\section{Introduction} \label{sec:intro}
In the last years digital measuring systems become the standard in all experimental sciences. By using \textit{virtual instruments} and new programable electronic
devices, such as Digital Signal Processors ($DSP$) and Field
Programmable Gate Arrays ($FPGA$)  experimenters may design and
modify their own measuring systems.

The effect of finite precision in these new devices needs to be
investigated. This issue is critical if  chaotic systems must be implemented, because due to roundoff errors digital implementations will always become periodic with a period $T$ and unstable orbits with a low period may become stable destroying completely the chaotic behavior.  
Grebogi and coworkers \cite{Grebogi1988} studied this subject and they shaw that the period $T$ scales with roundoff $\epsilon$ as
$T\sim\epsilon^{-d/2}$ where $d$ is the correlation dimension of
the chaotic attractor. 

To have a large period $T$ is one an important property of a chaotic map. Stochasticity and mixing are also relevant. Furthermore to characterize these properties several quantifiers were
studied  \cite{DeMicco2009}. Among them the use of an
entropy-complexity representation ($H-C$ plane) deserves special
consideration \cite{Rosso2007C,DeMicco2008,DeMicco2011,DeMicco2009,Rosso2009}. 
A fundamental issue is the criterium to select the distribution function ($PDF$) assigned to the time series. Causal and non causal options are possible. Here we consider the non-causal traditional $PDF$ obtained by normalization of the histogram of the time series. Its statistical quantifier is the normalized entropy$H_{hist}$ that is  a measure of equiprobability among all allowed values. We also consider a causal $PDF$  that is obtained by assigning ordering patterns to segments of trajectory of length $D$. This PDF were first proposed by Bandt \& Pompe in a seminal paper \cite{Pompe2002}. The corresponding entropy $H_{BP}$ was also proposed as a quantifier by Bandt \& Pompe. Amig\'o and coworkers proposed the number of forbidden patterns as a quantifier of chaos \cite{Amigo2007b}. Essentialy they reported the presence of forbidden patterns as an indicator of chaos. Recently it was shown that the name forbidden patterns is not convenient and it was replaced by  \textit{missing patterns }(MP) \cite{Rosso2012b}. 

Switching systems naturally arise in power electronics and many other areas in digital electronics. They have also interest in transport problems in deterministic ratchets \cite{Zarlenga2009} and it is known that synchronization of the switching procedure affects the output of the controlled system. Nagaraj et al \cite{Nagaraj2008} studied the case of switching between two maps. They shaw that the period $T$ of the
compound map obtained by switching between two chaotic maps is
higher than the period of each map.  Liu et al \cite{Liu2006} studied different switching rules applied to linear systems to generate chaos. Switching chaos was also addressed in \cite{Gluskin2008}.  Skipping values of the time series is another simple technique used to increase mixing in chaotic maps \cite{DeMicco2008}. 

In this paper we study the statistical characteristic of two well known maps: the tent map (TENT) and logistic map (LOG). Three additional maps are generated: 1) SWITCH, generated by switching between TENT and LOG; 2) EVEN, generated by skipping all the elements in odd position in SWITCH time series and 3) ODD, generated by discarding all the elements in an even position in SWITCH time series. Floating point, decimal numbers and binary numbers are used. All these specific numerical systems may be implemented in modern programmable logic boards. 

The main contributions of this paper are:
\begin{enumerate}
\item the definition of different statistical quantifiers and their relationship  with the properties of the time series generated by the map. 
\item the study of how this quantifiers are modified by the numerical representation using floating point, decimal and binary numbers. It is specially interesting to note that some systems (TENT) with very nice statistical properties in the world of the real numbers, become ``pathological" when numerical representations are used.
\item the effect of switching between two different maps, on the period and the statistical properties of the time series. Floating point, decimal and binary numerical representations are considered. 
\item the effect of skipping values in any of these maps
\end{enumerate}

Organization of the paper is as follows: section \ref{sec:quanti} describes the statistical quantifiers used in the paper and the relationship between their value and characteristics of the causal and non causal PDF considered; section \ref{sec:resultados} shows and discuss the results obtained for all the numerical representations. Finally section  \ref{sec:conclusions} deals with final remarks and future work. 